\section{TRABALHOS RELACIONADOS}

No cotidiano, um bom ponto de partida para se resolver um problema é procurar soluções já existentes para utilizá-las. Costumeiramente, as soluções que já existentes não se aplicam diretamente ao nosso caso, precisando ser adaptadas.

Assim, antes de se começar a resolver questões de pesquisa, é preciso conhecer o que tem de mais atual no seu tema.  Usando a abordagem de \citeonline{wazlawick2014metodologia} para explicar a necessidade de se conhecer trabalhos relacionados, cabe lembrar que antes de se construir uma nova ponte é importante conhecer os tipos de pontes que já existem; é preciso conhecer qual a atualidade do assunto estudado. Do contrário, pode estar construindo uma catapulta achando que se trata da melhor forma de atravessar um rio!

Para cada texto relacionado relevante encontrado, escreva: 1) qual a relação dele com seu trabalho, de que forma contribui; 2) que maneira a proposta se assemelha ao trabalho relacionado, ou seja, qual a relação direta entre os dois; 3) por fim, informa-se em que aspecto a proposta se difere do trabalho relacionado. Escreva de forma fluente, de maneira que não se perceba três fragmentos no texto.

A extensão e a profundidade necessária deste levantamento de trabalhos relacionados são determinados pelo perfil de sua área de conhecimento, e pelo seu orientador. Mas uma coisa é certa: não se pode dizer que seu trabalho é bom e justificável, se não houver como compará-lo a outros trabalhos que já existem.

\begin{alineas}

\item[a.] Corpo de Conhecimento: quando dela se utiliza conceitos já estabelecidos; este conteúdo que aparece mais destacadamente na seção Referencial teórico/revisão bibliográfica do seu trabalho;

\item[b.] Metodologia: alguns trabalhos são uma boa referência para o estabelecimento da metodologia de pesquisa; este conteúdo em geral subsidia a seção Procedimentos Metodológicos.

\item[c.] Trabalho relacionado: trabalhos que possuam mesma motivação, objetivo ou, em alguns casos específicos, metodologia. Ao se ler um bom trabalho relacionado, automaticamente surgem pensamentos como “ah, ele fez assim e posso fazer parecido” ou “não! esse aspecto do trabalho poderia ser melhor, prefiro fazer assim e assim”.  Se esses tipos de pensamento surgirem, então terá encontrado um bom texto candidato a ser considerado Trabalho Relacionado.

\end{alineas}

Algumas referências podem facilitar muito a sua busca por conhecer a atualidade do tema de estudo proposto, ajudando o pesquisador em diferentes aspectos do seu trabalho. Tipicamente, estes são os materiais denominados surveys (levantamentos), podendo ser compilações de:

\begin{alineas}

\item[d.] Estado-da-arte: artigos que apresentem conceitos mais recentes, estabelecidos na literatura científica;

\item[e.] Estado-da-prática: semelhante ao anterior, mas com foco no que está estabelecido atualmente como status quo da prática profissional.

\end{alineas}

Uma coisa é certa: enquanto o pesquisador não encontrar trabalhos relacionados à sua proposta, pode ter a certeza de que não procurou corretamente!